\documentclass[english,man]{apa6}

\usepackage{amssymb,amsmath}
\usepackage{ifxetex,ifluatex}
\usepackage{fixltx2e} % provides \textsubscript
\ifnum 0\ifxetex 1\fi\ifluatex 1\fi=0 % if pdftex
  \usepackage[T1]{fontenc}
  \usepackage[utf8]{inputenc}
\else % if luatex or xelatex
  \ifxetex
    \usepackage{mathspec}
    \usepackage{xltxtra,xunicode}
  \else
    \usepackage{fontspec}
  \fi
  \defaultfontfeatures{Mapping=tex-text,Scale=MatchLowercase}
  \newcommand{\euro}{€}
\fi
% use upquote if available, for straight quotes in verbatim environments
\IfFileExists{upquote.sty}{\usepackage{upquote}}{}
% use microtype if available
\IfFileExists{microtype.sty}{\usepackage{microtype}}{}

% Table formatting
\usepackage{longtable, booktabs}
\usepackage{lscape}
% \usepackage[counterclockwise]{rotating}   % Landscape page setup for large tables
\usepackage{multirow}		% Table styling
\usepackage{tabularx}		% Control Column width
\usepackage[flushleft]{threeparttable}	% Allows for three part tables with a specified notes section
\usepackage{threeparttablex}            % Lets threeparttable work with longtable

% Create new environments so endfloat can handle them
% \newenvironment{ltable}
%   {\begin{landscape}\begin{center}\begin{threeparttable}}
%   {\end{threeparttable}\end{center}\end{landscape}}

\newenvironment{lltable}
  {\begin{landscape}\begin{center}\begin{ThreePartTable}}
  {\end{ThreePartTable}\end{center}\end{landscape}}

  \usepackage{ifthen} % Only add declarations when endfloat package is loaded
  \ifthenelse{\equal{\string man}{\string man}}{%
   \DeclareDelayedFloatFlavor{ThreePartTable}{table} % Make endfloat play with longtable
   % \DeclareDelayedFloatFlavor{ltable}{table} % Make endfloat play with lscape
   \DeclareDelayedFloatFlavor{lltable}{table} % Make endfloat play with lscape & longtable
  }{}%



% The following enables adjusting longtable caption width to table width
% Solution found at http://golatex.de/longtable-mit-caption-so-breit-wie-die-tabelle-t15767.html
\makeatletter
\newcommand\LastLTentrywidth{1em}
\newlength\longtablewidth
\setlength{\longtablewidth}{1in}
\newcommand\getlongtablewidth{%
 \begingroup
  \ifcsname LT@\roman{LT@tables}\endcsname
  \global\longtablewidth=0pt
  \renewcommand\LT@entry[2]{\global\advance\longtablewidth by ##2\relax\gdef\LastLTentrywidth{##2}}%
  \@nameuse{LT@\roman{LT@tables}}%
  \fi
\endgroup}


\ifxetex
  \usepackage[setpagesize=false, % page size defined by xetex
              unicode=false, % unicode breaks when used with xetex
              xetex]{hyperref}
\else
  \usepackage[unicode=true]{hyperref}
\fi
\hypersetup{breaklinks=true,
            pdfauthor={},
            pdftitle={An Extension of the QWERTY Effect: Not Just the Right Hand, Expertise and Typability Predict Valence Ratings of Words},
            colorlinks=true,
            citecolor=blue,
            urlcolor=blue,
            linkcolor=black,
            pdfborder={0 0 0}}
\urlstyle{same}  % don't use monospace font for urls

\setlength{\parindent}{0pt}
%\setlength{\parskip}{0pt plus 0pt minus 0pt}

\setlength{\emergencystretch}{3em}  % prevent overfull lines

\ifxetex
  \usepackage{polyglossia}
  \setmainlanguage{}
\else
  \usepackage[english]{babel}
\fi

% Manuscript styling
\captionsetup{font=singlespacing,justification=justified}
\usepackage{csquotes}
\usepackage{upgreek}

 % Line numbering
  \usepackage{lineno}
  \linenumbers


\usepackage{tikz} % Variable definition to generate author note

% fix for \tightlist problem in pandoc 1.14
\providecommand{\tightlist}{%
  \setlength{\itemsep}{0pt}\setlength{\parskip}{0pt}}

% Essential manuscript parts
  \title{An Extension of the QWERTY Effect: Not Just the Right Hand, Expertise
and Typability Predict Valence Ratings of Words}

  \shorttitle{QWERTY EFFECT EXTENSION}


  \author{Erin M. Buchanan\textsuperscript{1}~\& Kathrene D. Valentine\textsuperscript{2}}

  \def\affdep{{"", ""}}%
  \def\affcity{{"", ""}}%

  \affiliation{
    \vspace{0.5cm}
          \textsuperscript{1} Missouri State University\\
          \textsuperscript{2} University of Missouri  }

  \authornote{
    \newcounter{author}
    Erin M. Buchanan is an Associate Professor of Quantitative Psychology at
    Missouri State University. K. D. Valentine is a Ph.D.~candidate at the
    University of Missouri.

                      Correspondence concerning this article should be addressed to Erin M. Buchanan, 901 S. National Ave, Springfield, MO 65897. E-mail: \href{mailto:erinbuchanan@missouristate.edu}{\nolinkurl{erinbuchanan@missouristate.edu}}
                          }


  \abstract{Typing is a ubiquitous daily action for many individuals; yet, research
on how these actions have changed our perception of language is limited.
The QWERTY effect is an increase in valence ratings for words typed more
with the right hand on a traditional keyboard (Jasmin \& Casasanto,
2012). Although this finding is intuitively appealing given both right
handed dominance and the smaller number of letters typed with the right
hand, extension and replication of the right side advantage is
warranted. The present paper reexamined the QWERTY effect within the
embodied cognition framework (Barsalou, 1999) and found that the right
side advantage is replicable to new valence stimuli, as well as
experimental manipulation. Further, when examining expertise, right side
advantage interacted with typing speed and typability (i.e.~alternating
hand keypresses or finger switches) portraying that both skill and our
procedural actions play a role in judgment of valence on words.}
  \keywords{keyboard, valence, QWERTY, word norms \\

    
  }





\usepackage{amsthm}
\newtheorem{theorem}{Theorem}
\newtheorem{lemma}{Lemma}
\theoremstyle{definition}
\newtheorem{definition}{Definition}
\newtheorem{corollary}{Corollary}
\newtheorem{proposition}{Proposition}
\theoremstyle{definition}
\newtheorem{example}{Example}
\theoremstyle{definition}
\newtheorem{exercise}{Exercise}
\theoremstyle{remark}
\newtheorem*{remark}{Remark}
\newtheorem*{solution}{Solution}
\begin{document}

\maketitle

\setcounter{secnumdepth}{0}



From its creation in 1868, to its appearance in our homes today, the
QWERTY keyboard has held the interest of psychologists. The process of
typing on a keyboard requires many procedures to function in tandem,
which creates a wealth of actions to research (Inhoff \& Gordon, 1997).
Rumelhart and Norman's (1982) computer model of skilled typing is still
highly influential. They hypothesize that typing results from the
activation of three levels of cognition: the word level, the keypress
level, and the response level. They believe that after word perception,
the word level is activated, causing the keypress level to initiate a
schema of the letters involved in typing the word. This schema includes
the optimal position on the keyboard for that specific hand-finger
combination to move to at the appropriate time for individual
keystrokes. Concurrently, the response system sends feedback information
to initiate a keypress motion when the finger is in the appropriate
space. Their theory proposes that schemata and motion activations occur
simultaneously, constantly pulling or pushing the hands and fingers in
the right direction.

While many studies have focused on errors in typing to investigate
response system feedback (F. A. Logan, 1999), G. D. Logan (2003) argued
for parallel activation of keypresses. He examined the Simon effect to
show more than one letter is activated at the same time, and
consequently, the second keypress motion is begun before the first
keypress is done. The Simon effect occurs when congruent stimuli create
faster responses than incongruent stimuli, much like the Stroop task
(Simon, 1990; Simon \& Small, 1969). For example, if we are asked to
type the letter f (a left handed letter), we type it faster if the f is
presented on the left side of the screen. Similarly, Rieger (2004)
reported finger-congruency effects by altering a Stroop task:
participants were required to respond to centrally presented letters
based on color-key combinations. When the letter and color were
congruent (i.e.~a right-handed letter was presented in the designated
color for a right response), the skilled typists' responses were faster
than incongruent combinations. Further, this effect was present when
participants responded to items with their hands crossed on the
responding device, suggesting the effect was expertise-based rather than
experiment-response based. These results imply that automatic actions
stimulate motor and imagery representations concurrently and may be
linked together in the brain (Hommel, Müsseler, Aschersleben, \& Prinz,
2001; G. D. Logan \& Zbrodoff, 1998; Rieger, 2004). This dual activation
of motor and imagined items is the basis for embodied cognition, a
rapidly expanding field in psychology (Barsalou, 1999; Salthouse, 1986).

\subsection{Embodied Cognition}\label{embodied-cognition}

While the mind was traditionally considered an abstract symbol processor
(Newell \& Simon, 1976), newer cognitive psychology theories focus on
the interaction between the brain's sensorimotor systems and mental
representations of events and objects (Barsalou, 1999; Zwaan, 1999). The
interplay between these systems has been found in both neurological
(Hauk, Johnsrude, \& Pulvermüller, 2004; Lyons et al., 2010; Tettamanti
et al., 2005) and behavioral research (Cartmill, Goldin-Meadow, \&
Beilock, 2012; Holt \& Beilock, 2006; Zwaan \& Taylor, 2006). Motor
representations of tasks are activated even when not specifically asked
to perform the task, and if the action is well-learned, the task is
perceived as pleasant (Beilock \& Holt, 2007; Ping, Dhillon, \& Beilock,
2009; Yang, Gallo, \& Beilock, 2009). For example, Beilock and Holt
(Beilock \& Holt, 2007) asked novice and expert typists to pick which
one of two letter dyads they preferred, which were either different hand
combinations (CJ) or same finger combinations (FV). They found that
novices have no preference in selection, while expert typists more
reliably picked the combinations that were easier to type. To show that
this effect was due to covert motor representation activation, and thus,
expanding on findings from Van den Bergh, Vrana, and Eelen (1990),
participants also made preference selections while repeating a keypress
combination. When expert motor planning was distracted by remembering
the pattern presented, no preference for letter dyads was found,
indicating that the simultaneous activation of the motor representation
was necessary to influence their likability ratings. Similar embodied
findings have also been portrayed with emotionally charged sentences and
facial movements (Havas, Glenberg, \& Rinck, 2007), positive-negative
actions, such as head nodding or arm movements (Glenberg, Webster,
Mouilso, Havas, \& Lindeman, 2009; Ping et al., 2009), and
perceptuomotor fluency (Oppenheimer, 2008; Yang et al., 2009).

\subsection{Body Specificity
Hypothesis}\label{body-specificity-hypothesis}

Using an embodied framework, Casasanto (2009) has proposed that
handedness dictates preference because our representations of actions
are grounded in our physical interactions with the environment. In
several studies, he portrayed that handedness influenced preference for
spatial presentation (i.e.~left handed individuals associate
\enquote{good} with left, while right handed individuals associate
\enquote{good} with right), which in turn influenced judgments of
happiness and intelligence and our decision making in hiring job
candidates and shopping. In all these studies, participants reliably
selected the hand-dominant side more often, which does not match
cultural or neurolinguistic representations of positive-is-right and
negative-is-left (Davidson, 1992). These findings imply that our
handedness is a motor expertise that causes ease of action on the
dominant side to positively influence our perceptions of items presented
on that side. Further, Casasanto (2011) compiled a review of body
specific actions and their representation in the brain using fMRIs.
Handedness interacted with imagining actions, reading action, and
perceiving the meanings of action verbs, such that fMRI patterns were
mirrored for left and right handed participants matching their dominant
side.

\subsection{The QWERTY Effect}\label{the-qwerty-effect}

These effects inspired Jasmin and Casasanto (2012) to propose the idea
that typing, an action that often replaces speaking, has the ability to
create semantic changes in how we perceive words. The asymmetrical
arrangement of letters on the QWERTY keyboard increases fluency of
typing letters on the right side because there are fewer keys, and thus,
less competition for fingers. That arrangement should then cause us to
perceive the letters on the right side as more positive and letters on
the left side as more negative. Consequently, words that are composed of
more letters from the right side (the right side advantage; RSA) should
be rated as more positive than those with more letters on the left. They
found this preference for RSA over three languages (English, Spanish,
and Dutch), and the effect was even stronger on words created after the
invention of the QWERTY keyboard (i.e.~lol), as well as evident in
pseudowords such as plook. However, in contrast to the body specificity
hypothesis, left and right handed participants showed the same trend in
effects for positive-is-right words.

\section{Current Study}\label{current-study}

The current study examined the right side advantage's interaction with
traditional embodied cognition definitions (expertise, fluency). We
analyzed the different implications of the body specificity hypothesis
and a more general embodied hypothesis by testing the following: 1) To
examine embodied cognition, we coded each word for number of hand
alternations (akin to Beilock and Holt's (2007) different hand
preferences). Given that typing involves the procedural action system,
we would also expect to find that increased hand switches are positively
related to ratings of valence because words that are typed on
alternating hands are easier to type. 2) The interaction between RSA and
switches was examined to determine if these hypotheses can be combined
(i.e.~we only like right handed words because we have to switch back and
forth to type the more commonly used letters, such as \emph{e} or
\emph{a}).

\section{Experiment 1}\label{experiment-1}

\section{Method}\label{method}

\subsection{Participants}\label{participants}

Participants (\emph{N} = 546) were recruited from the university
undergraduate human subject pool and received course credit for their
time. 65233 rows of data were present for these participants, where 504
participants included complete data (i.e.~120 rows, see below), 39 were
missing one data point, and 3 were missing many data points. All data
points were included, and missing data points were usually computer
error (i.e.~freezing during the experiment) or participant error
(i.e.~missed key press).

Rating data were screened for multivariate outliers, and one
participant's ratings were found to have extreme Mahalanobis distance
scores (Tabachnick \& Fidell, 2012) but were kept in the data set. 11.5
percent of the sample was left-handed, 0.2 marked ambigdextrious, and
0.4 was missing handedness information. The average typing speed was
48.20 (\emph{SD} = 13.45, and the average percent accuracy rate for the
typing test was 92.59 (\emph{SD} = 8.63).

\subsection{Materials}\label{materials}

Both Experiment 1 and Experiment 2 use the English ANEW (Bradley \&
Lang, 1999) norms to create stimuli for this study, in an effort to
replicate Jasmin and Casasanto's (2012) experiments, and 240 words were
selected for this experiment (120 real words, 120 pseudowords).
Pseudowords were selected from Appendix E of the supplementary materials
presented from the QWERTY publication. These words were coded as
described below for RSA, switches, word length, and letter frequency.
Average word length was 4.85 (\emph{SD} = 1.51; range = 3 - 13).

\subsection{Coding}\label{coding}

Each of the words used in this experiment and Experiment 2 were coded
for control and experimental variables. Control variables included word
length and average letter frequency. Average letter frequency was
created by averaging the English letter frequency (Lewand, 2000) for
each letter in a word. Words with high average letter frequencies
contain more commonly used letters (\emph{e, t, a, o}); while words with
lower frequencies use more of the less common letters (\emph{z, q, x,
j}). Experimental variables included RSA, number of hand switches, and
number of finger switches. Typing manuals were consulted, and letters
were coded as left (\emph{q, w, e, r, t, a, s, d, f, g, z, x, c, v, b})
or right-handed letters (\emph{y, u, i, o, p, h, j, k, l, n, m}). Left
handed letters were coded with -1 and right handed letters with +1,
which created summed scores indicating the overall right side advantage
for a word. Words were coded for the number of hand switches within a
word using the left-right coding system described above. Finally, the
number of finger switches were coded using traditional typing manuals
for each finger. Finger switches was highly correlated with word length,
\emph{r} = .89, and therefore, word length was excluded as a control
variable due the interest in typing skill for experimental hypotheses.

\subsection{Procedure}\label{procedure}

Upon consent to participate in the experiment, participants were given a
typing test by using a free typing test website (TypingMaster, Inc.,
2013). Each participant typed Aesop's Fables for one minute before the
website would reveal their typing speed and accuracy rate, which was
recorded by the experimenter. After this test, participants indicated
their dominant writing hand. Participants were then given 120 of the 240
stimuli to rate for pleasantness (60 real words, 60 pseudowords). This
smaller number of stimuli was used to control fatigue/boredom on
participants. These stimuli were counterbalanced across participants,
and the order of the stimuli was randomized. Participants were told to
rate each word for how pleasant it seemed using a 9 point Likert type
scale (1 - \emph{very unpleasant}, 4 - \emph{neutral}, 9 - \emph{very
pleasant}). The same self-assessment manikin from Jasmin and Casasanto
(2012) was shown to participants at the top of the computer screen to
indicate the points on the Likert scale. The words appeared in the
middle of the screen in 18 point Arial font. Participants then typed the
number of their rating on the computer keyboard. Once they rated all
stimuli, participants were debriefed and allowed to leave.

\section{Results}\label{results}

\subsection{Data Analytic Plan}\label{data-analytic-plan}

Because each participant constituted multiple data points within the
dataset, a multilevel model was used to control for correlated error
({\textbf{???}}). ({\textbf{???}})'s \emph{nlme} package in \emph{R} was
used to calculate these analyses. A maximum likelihood multilevel model
was used to examine hypotheses of interactions between typing speed,
hand/finger switching, and RSA while controlling for letter frequency
predicting item pleasantness ratings. Participants were included as a
random intercept factor. Typing speed, finger/hand switches, and RSA
were mean centered before analyses to control for multicollinearity.

\subsection{Main Effects}\label{main-effects}

\subsection{Interactions}\label{interactions}

\section{Experiment 2}\label{experiment-2}

\section{Method}\label{method-1}

\subsection{Participants}\label{participants-1}

Similar to Experiment 1, sixty participants were recruited from the
university undergraduate human subject pool and received course credit
for their time. Rating data were screened for multivariate outliers.
Again, one participant's ratings were found to have extreme Mahalanobis
distance scores (Tabachnick \& Fidell, 2012). However, this individual's
ratings were left in the data set. Approximately 8 percent of the sample
was left-handed. The average percent accuracy rate for the typing test
was 93.58 (\emph{SD} = 5.26).

\subsection{Procedure}\label{procedure-1}

While materials and coding were the same for Experiments 1 and 2,
procedure for Experiment 2 differed slightly. In this study, when
participants were shown the word (or pseudoword) on the screen, they
were first asked to type the word on the keyboard in front of them.
After they had typed the word, they were then asked to rate the word for
pleasantness using the scale and self-assessment manikin discussed
previously.

\section{Results}\label{results-1}

\subsection{Main Effects}\label{main-effects-1}

\subsection{Interactions}\label{interactions-1}

\section{Discussion}\label{discussion}

YADA SCHMADA CHANGE THIS SECTION These results imply that the QWERTY
keyboard has influenced our perceptions of words, in a more complex way
than a simple body specificity hypothesis. In the overall normed
database analyses, the original QWERTY effect was replicable across a
large body of various types of stimuli (verbs, Twitter, category norms),
with much the same size of effect as Jasmin and Casasanto (2012)
published. Word length was often negatively related to valence ratings,
which indicated that we like shorter words to type. Average letter
frequency was usually a positive predictor of valence ratings wherein
ratings are higher for words with more frequent letters; however, these
effects were inconsistent. Our measure of fluency (switches) varied
across stimulus sets but it appears, by analyzing multiple sources of
ratings for words at the same time, that there might have been an
interaction between RSA and number of switches. This interaction
portrayed that we find words that switch off of left-handed keypresses
as more pleasant, while right-handed keypresses are preferable by
switching hands less often. These effects were examined in more detail
in Experiment 2, which incorporated Beilock and Holt's (2007) study by
including typing speed as a measure of expertise. Word ratings turned
out to be quite complex with a four-way interaction between
real/pseudowords, switches, RSA, and typing speed. All analyses showed a
positive effect of right-side words, as well as if they were shorter and
used more frequent letters. However, for pseudowords, no other effects
were significant. Both Beilock and Holt (2007) and Van der Bergh et al.
(1990) showed expert preferences for two and three letter combinations
that were typed with different fingers. Our results could imply that our
embodied actions influence preferences for procedures that are more
likely in our environment. While our pseudowords were legal English
phoneme combinations, they are extremely unlikely to have been
previously practiced or encountered in our daily tasks. Therefore,
switching preference will not extend to pseudowords (unpracticed
actions) because they are not fluent (Oppenheimer, 2008).

The effect of expertise was shown on real words, where the three-way
interaction between RSA, switches, and typing speed was examined by
separating out right, equal, and left-handed words. For right-handed
words, typing speed (or the interaction) was not a significant predictor
of valence, and while not significant, number of switches was negatively
related to valence ratings. For equally right-left and left-handed
words, pleasantness ratings increase by switching back and forth to the
right hand. Further, left-handed words showed an interaction between our
two embodied cognition variables, where the number of switches increases
valence ratings as the typing speed of the participant decreases.
Therefore, it appears that as participants gain fluency through
increased typing speed, the number of switches back and forth for
left-handed words matters less for pleasantness ratings. Many of the
most frequent letters on the QWERTY keyboard are on the left side, which
may frustrate a slow typist because of the need to coordinate finger
press schemata that involve same finger muscle movements (Rumelhart \&
Norman, 1982). Consequently, the number of switches becomes increasingly
important to help decrease interference from the need to continue to use
the same hand. The ease of action by switching back and forth is then
translated as positive feelings for those fluent actions (Oppenheimer,
2008).

These embodied results mirror a clever set of studies by Holt and
Beilock (2006) wherein they showed participants sentences that matched
or did not match a set of pictures (i.e.~the umbrella is in the air
paired with a picture of an open umbrella). Given dual-coding theory
(Paivio, 1971), it was not surprising that participants were faster to
indicate picture-sentence matches than non-matches (also see Stanfield
\& Zwaan, 2001; Zwaan, Stanfield, \& Yaxley, 2002). Further, they showed
these results extended to an expertise match; hockey and football
players were much faster for sentence-picture combinations that matched
within their sport than non-matches, while novices showed no difference
in speed for matches or non-matches on sports questions. Even more
compelling are results that these effects extend to fans of a sport and
are consistent neurologically (i.e.~motor cortex activation in experts;
Beilock \& Lyons, 2008). These studies clearly reinforce the idea that
expertise and fluency unconsciously affect our choices, even when it
comes to perceived pleasantness of words.

This extension of the QWERTY effect illuminates the need to examine how
skill can influence cognitive processes. Additionally, typing style,
while not recorded in this experiment, could potentially illuminate
differences in ratings across left-handed and right-handed words.
Hunt-and-peck typists are often slower than the strict typing manual
typists, which may eliminate or change the effects of RSA and switches
since typists may not follow left or right hand rules and just switch
hands back and forth regardless of key position. The middle of a QWERTY
layout also poses interesting problems, as many typists admit to
\enquote{cheating} the middle letters, such as t, and y or not even
knowing which finger should actually type the b key. Further work could
also investigate these effects on other keyboard layouts, such as
Dvorak, which was designed to predominately type by alternating hands to
increase speed and efficiency (Noyes, 1988).

\newpage

\section{References}\label{references}

\setlength{\parindent}{-0.5in} \setlength{\leftskip}{0.5in}

\hypertarget{refs}{}
\hypertarget{ref-Inhoff1997}{}
Inhoff, A. W., \& Gordon, A. M. (1997). Eye Movements and Eye-Hand
Coordination During Typing. \emph{Current Directions in Psychological
Science}, \emph{6}(6), 153--157.
doi:\href{https://doi.org/10.1111/1467-8721.ep10772929}{10.1111/1467-8721.ep10772929}






\end{document}
